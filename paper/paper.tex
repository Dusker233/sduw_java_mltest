\documentclass[a4paper, 11pt, cn]{elegantpaper}
\usepackage{ctex}
\usepackage{amssymb}
\usepackage{array}
\usepackage{xcolor}
\usepackage{cancel}
\usepackage{color}
\usepackage{tikz}
\usepackage{minted}
\usepackage{algorithmic}
\usepackage{algorithm2e}
\usepackage{xeCJK}
\usepackage{hyperref}
\usepackage{amsmath, amsthm}
\usepackage{mathrsfs}

\title{Java}
\author{禚文宇 \\ 202100800179}
\date{\today}

\begin{document}

\maketitle

\begin{abstract}
    
\end{abstract}

\section{数据集介绍}
本次上机作业使用的是 3w 数据集,3w 数据集是第一个公开的记录了油井中罕见的不良真实事件的数据集,可以作为基准数据集,用于开发与实际数据固有困难相关的机器学习技术。

关于该数据集背后的理论的更多信息,可在《石油科学与工程杂志》(Journal of Petroleum Science and Engineering)上发表的论文《油井中罕见不良真实事件的现实和公共数据集》\cite{3w}中找到。

对于数据集中的数据,其部分属性信息如下表所示:

\begin{table}[htbp]\centering
    \begin{tabular}{|c|c|}
    \hline
    属性    & 含义         \\ \hline
    P-PDG & 永久井下压力表的压力 \\ \hline
    P-TPT & 压力传感器的数据   \\ \hline
    T-TPT & 温度传感器的数据   \\ \hline
    \end{tabular}
    \caption{数据部分属性信息}
\end{table}

\section{使用的算法介绍}

本次上机实验,我尝试了两个分类算法,分别为K-最近邻(KNN)算法和朴素贝叶斯算法,下面我将简单介绍这两个算法。

\subsection{K-最近邻算法(KNN)介绍}

K-最近邻算法(KNN)是一种用于分类和回归的非参数统计方法:
\begin{enumerate}
    \item 在 KNN-分类中,通过 $K$ 个最近邻居中出现次数最多的分类决定了此对象的分类;
    \item 在 KNN-回归中,$K$ 的最近邻居的值的平均值将会称为此对象的预测值。
\end{enumerate}

KNN 是一个非常简单的机器学习算法,分为计算距离、取 $K$ 个最近邻居、根据邻居分类三个步骤。

\subsubsection{具体过程}

\begin{enumerate}
    \item 计算距离:在 KNN 中,我们通过 Euclid 距离来度量两个对象 $\theta_0=\left(x_0, x_1, \cdots, x_n\right)$ 和 $\theta_1 = \left(y_0, y_1, \cdots, y_n\right)$
之间的距离,具体定义为
$$\operatorname{dis}(\theta_0, \theta_1) = \sum\limits_{i=0}^{n}\sqrt{\left(x_i - y_i\right)^2}$$

    必须注意到的是,两个对象必须具有相同的数据维度,否则 Euclid 距离将无法计算。
    \item 取 $K$ 个最近邻居:本步骤非常容易,即按照 Euclid 距离排序后,取前 $K$ 个互异的数据点即可。
    \item 在 根据邻居分类:KNN-分类中,我们只需取出这 $K$ 个最近邻居的标签,找出出现次数最多的标签即为返回值。
\end{enumerate}


\subsection{朴素贝叶斯算法介绍}

朴素贝叶斯算法是一个基于贝叶斯公式的算法:设 $(\Omega, \mathscr{F}, P)$ 是概率空间,$A_1, A_2, \dots, A_n$ 是样本空间 $\Omega$ 的一个分割,则对任意 $B \in \mathscr{F}$,$P(B) > 0$,有
$$P\left(A_k \mid B\right) = \dfrac{P(A_k)P(B \mid A_k)}{\sum\limits_{j = 1}^nP(A_j)P(B \mid A_j)}$$ 我们可以这样理解这个公式:假设某个过程具有 $A_1, A_2, \cdots, A_n$ 这样 $n$ 个可能的前提(原因),而 $P(A_1), P(A_2), \cdots, P(A_n)$ 是
人们对这 $n$ 个可能的前提(原因)的可能性大小的一种事前估计,称之为\textbf{先验概率}。当这个过程有了一个结果 $B$ 之后,人们会通过条件概率 $P(A_1 \mid B), P(A_2 \mid B), \cdots, P(A_n \mid B)$ 来对这 $n$ 个可能前提的可能性大小做出一个新的认识,
因此将这些条件概率称之为\textbf{后验概率},而贝叶斯公式恰好提供了一种计算后验概率的工具。



\section{实验过程}

\subsection{数据清洗}

\subsection{min-max 标准化}

\subsection{使用 Java-ML 库中自带的 KNN 算法}

\subsection{手写 KNN 算法}

\subsection{使用 Java-ML 库中自带的朴素贝叶斯算法}

\appendix

\nocite{*}
\printbibliography[heading=bibintoc, title=\ebibname]
\end{document}